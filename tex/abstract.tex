本研究の目的は、豊富なメモリ資源を持たないマイクロコントローラ上で動作させることが可能な、
記述性に優れた独自の関数リアクティブプログラミング(FRP)言語の設計手法を提案することである。
FRPは入出力の相互作用を宣言的に記述することによってソフトウェアの動作を表現するプログラミング手法であり、
非同期的に発生するイベントに反応し何らかの処理を行うプログラムを簡潔に記述することができる。
GUIアプリケーションのフロントエンド開発等の領域では既に広く実用化されている一方で、
小規模組込みシステム開発等の計算資源に制約を抱える領域においては、現在までにあまり実用化されて来なかった。
これは、従来のFRP言語やFRPフレームワークは豊富なメモリ資源を活用する形で実現されており、
マイクロコントローラ等の豊富なメモリ資源を持たない環境において適用することは難しいためである。

本研究では、FRPの優れた表現性能は必ずしも豊富なメモリ資源の活用を前提として成り立っているわけでは無いことを明らかにする
ために、僅かなメモリ使用量で動作する独自のFRP言語{\it SFRP}を設計及び実装した。
SFRPはメモリ制約環境下において避けるべき動的メモリ確保を完全に排したプログラミング言語でもあり、
C言語コンパイラがサポートされるマイクロコントローラ上で動作させることができる。

本論文では、SFRPが採用するFRPの計算モデルを示し、如何にして省メモリ性を実現しているのかを述べると共に、
SFRPとマイクロコントローラを用いた実用的な小規模組込みシステムの開発手順についても例示する。
同時にSFRPの空間的・計算量的な性能評価についても行い、記述性と実用性についても議論する。
