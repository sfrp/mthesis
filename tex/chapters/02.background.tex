\chapter{背景知識}
\section{関数リアクティブプログラミング}
入出力の相互作用を関数によって表現するプログラミング手法を、
関数リアクティブプログラミング(Functional Reactive Programming, FRP)と呼ぶ。
FRPでは、時刻に依存して変動する値を時変値(time-varying value)として抽象化し、
時変値どうしを組み合わせて新たな時変値を定義することによってプログラムを表現する。
すなわち、アプリケーションの入力値に相当する時変値を予め用意しておき、
それを組み合わせることによって出力に相当する時変値を定義し、アプリケーションの動作を表現するのである。

FRPは1997年にConal Elliott\cite{elliott1997functional}らによって提唱され、
彼らによって初のFRPフレームワークFranが開発された。
Franは純粋関数型言語であるHaskellにおける内部DSLとして実装され、表現の自由度こそ高いものの、
実行性能には問題があった。
memory-leakやtime-leakを引き起こす問題、
すなわち実行時にメモリ使用量や計算所要時間が無限に増大していく状況が発生する問題を抱えていたのである。
その後、Franが抱える問題を解消するために新たなFRP手法の研究が行われた。
FRPの表現性能と実行性能とを天秤に掛けながら、
Real-Time-FRP\cite{wan2001real}やArrowized-FRP\cite{nilsson2002functional}、
Event-Driven-FRP\cite{simulinkwebsite}
といった種々のFRP手法が提唱されていったのである。
これらのFRP手法はFranが当初持っていた自由な表現性能からはやや乖離しつつも、
memory-leakやtime-leakを解消することに成功した。
また2015年にはFranに近い表現性能を保ちつつmemory-leackやtime-leakを抑えたFRPフレームワーク\cite{ploeg2015practical}が発表され、
現在ではFranを始祖とするHaskellを前提としたFRPの研究はある程度の収まりを見せている。

一方で2009年にはリアクティブプログラミングのためのReactiveX(Reactive Extensions, Rx)ライブラリがMicrosoft社によって発表された。
当初は.NET Frameworkに向けて作られたRx.NETのみが提供されていたが、現在では様々な言語向けに移植され開発が続けられている。
ReactiveXは参照透明性こそ保証しないが、抽象化の手法はHaskellにおいて研究されたEvent-Driven-FRPなどに近いものがある。

2012年にはWebフロントエンドアプリケーションの記述に特化したFRP言語Elm\cite{czaplicki2012elm,czaplicki2013asynchronous}
が発表された。
過去にHaskellを前提として研究されたFRPの手法を取り入れつつ、
Haskellに依らない独自の純粋関数型言語として実装されたものである。
ElmはFRPの手法をHaskellの外の世界に持ち出すことに成功したという点で大きな影響力を示した。
また現在でも精力的に開発が続けられており、Webアプリケーションの開発に広く実用されている。


\section{マイクロコントローラ}
コンピュータシステムを1つの集積回路としてコンパクトに纏めたものを、マイクロコントローラと呼ぶ。
CPUやメモリ、入出力回路やタイマー回路などが一纏めにされており、電源を繋ぐだけで動作可能なものも多い。
現在では家電を始めとする電子機器の大半にマイクロコントローラが搭載され、その機能制御を担っている。

様々な用途向けのマイクロコントローラが製品化されており、製品によってその価格や性能は様々であるが、
低価格帯のマイクロコントローラの多くは数KBのFlashとその1/10程度のRAMを持つ程度であり、豊富なメモリ資源を持たない。
アセンブリ言語を用いる程ではないにしても、C++の高レベルな機能を利用することは難しい場合も多い。
C++によって記述されたライブラリをリンクするとプログラムサイズが大きくなりがちなためである。
また、マイクロコントローラによっては対応するC++コンパイラが存在しないかあるいは高価で手軽に利用できない場合もある。

一般に、小規模組込みシステム開発においては、動的にメモリを確保することは避けた方がよい。
プログラムの動作中にメモリが枯渇してしまうとプログラムを続行することが困難になり、機器が制御不能に陥るためである。
関数の再帰呼出しはスタック領域におけるメモリの動的確保であるため、これも避けた方がよい。
C++のSTLを始めとする、暗黙的にメモリの動的確保を行うライブラリには注意が必要である。

電子機器の中で独立して動作するシステムにとって、プログラムが正常に動作することの保証は何よりも大切である。
したがって、バグを作りこまないための記述性やテスト可能性を確保することが肝要となる。
