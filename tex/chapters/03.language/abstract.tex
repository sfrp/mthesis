一般に、反応的なプログラムのある時点における出力値は、それ以前の時点において受け取った入力値に依存して決まる。
したがって、この依存関係を記述することによって反応的なプログラムを表現することができる。
この依存関係を記述するために、SFRPでは時変値(time-varying value)と呼ばれる抽象概念を用いる。
時変値とは過去のFRP研究において提唱された概念であり、入出力に代表されるような時刻によって値が変動する存在を指す言葉である。
SFRPでは、この時変値の依存関係を定義することによってプログラムを表現する。
