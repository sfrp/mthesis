\subsection{時変値によるプログラムの表現}\label{sec:language:model:program}
本項では、SFRPにおいて時変値がどのように表現され、ユーザはどのような形で時変値を定義することができるのかを述べていく。

\begin{itemize}
  \item
  時変値は時刻をパラメータに取る関数とみなされる。
  \begin{equation*}
    Time \rightarrow Value
  \end{equation*}

  \item
  プログラムへの入力は、プリミティブな時変値
  \footnote{他の時変値を用いて定義されるのではなく、前提として与えられる時変値をプリミティブな時変値と呼ぶ。}
  として表現される。
  \begin{equation*}
    {\it input} \, :: \, Time \rightarrow Value
  \end{equation*}

  \item
  他の時変値を参照して演算を加えることにより、新たな時変値を定義することができる。
  たとえば、時変値Aの値と時変値Bの値の和を取る時変値Cを定義することができる。
  \begin{equation*}
    C(t) := A(t) + B(t)
  \end{equation*}

  \item
  各時変値は、時刻$t$における値(現在値)の参照と、微小時間$ \Delta(t) $前の値(直前値)の参照のみが許されており、
  任意の時刻における値の参照が許されているわけではない。
  たとえば、時変値Aの現在値と時変値Bの直前値の和を取る時変値Dを定義することができる。
  \begin{equation*}
    D(t) := A(t) + B(t - \Delta(t))
  \end{equation*}

  \item
  $ \Delta $は微小時間を表すプリミティブな時変値である。
  \begin{equation*}
    \Delta(t) > 0
  \end{equation*}

  \item
  初期値($ t < t_0 $における値)を特別に指定した時変値を定義することができる。
  たとえば、$t \geq t_0$においては時変値Aの値を取り、$t < t_0$においては0を取る時変値Bを定義することができる。
  ただし、$t_0$は任意の定数である。
  \begin{equation*}
    B(t) := \begin{cases}
      A(t) & (t \geq t_0) \\
      0 & (t < t_0)
    \end{cases}
  \end{equation*}

  \item
  プログラムの出力は、任意の時変値によって定義される。
  以下は、時変値Dの現在値に1を加えた時変値を出力とする例である。
  \begin{equation*}
    output(t) := D(t) + 1
  \end{equation*}
\end{itemize}

\subsection{時変値プログラムの実行}\label{sec:language:model:execution}
\ref{sec:language:model:program}項にて示した表現は、入力時変値を前提とした出力時変値の定義である。
これを実行するということはすなわち、入力値の取得と出力値の計算・反映を順に行う処理の反復であり、
例えば以下に示す逐次操作によって実現することができる。
\begin{screen}
\begin{enumerate}
  \item $ {\it timestamp}^{\prime} := 現在時刻$
  \item $ {\it timestamp} := 現在時刻 (> timestamp^{\prime})$
  \item $ {\it elapsed} := {\it timestamp} - {\it timestamp}^{\prime} $
  \item $\tau$が未定義であるなら$ \tau := 0 $として、そうでないなら$ \tau := \tau + {\it elapsed} $とする。
  \item $ \Delta(\tau) := {\it elapsed} $
  \item 入力を受け取り、$ {\it input}(\tau) $の値とする。
  \item プログラム中の全ての時変値の、$t=\tau$における値を求める。
  \item $ {\it output}(\tau) $の値を出力する。
  \item $ {\it timestamp}^{\prime} := {\it timestamp} $
  \item 2に戻る。
\end{enumerate}
\end{screen}
%
% 以下では、この逐次操作が\ref{sec:language:model:program}項にて示した時変値プログラムに対して、
% 失敗すること無く適用可能であるかどうかを論じる。
% \ref{sec:language:model:execution}項の操作において失敗する可能性が考えられるのは、7番目の処理(処理7)である。
% ただし入出力の処理と、数値に対する演算処理は必ず成功すると仮定する。
%
% ${\it output}(\tau)$の値の計算過程において、存在しない時変値が参照される場合は当然の事ながら処理7は実行不可能である。
% 以下では、存在しない時変値への参照はプログラム中に含まれないと仮定する。
% また、時変値の参照による依存関係に循環は無いと仮定する。
% ただし時変値Aが時変値Bの現在値を参照して定義されているときに限り、AはBに依存すると見なす。
%
% $\tau = a(\neq 0)$において処理7が実行される場合、
% 過去の$\tau = a - \Delta(a)$である場合において処理7の実行に成功しているはずである。
% したがって、プログラム中の任意の時変値$v$について、直前値である$v(\tau - \Delta(\tau))$は計算可能である。
% このことから補題1が成り立つ。
% \begin{itembox}[l]{補題1}
%   $\tau = 0$のとき処理7の実行に成功する $\Longleftrightarrow$ 常に処理7の実行に成功する
% \end{itembox}
%
% $\tau < 0$において処理5および処理6が行われることはないため、
% 任意の$a < 0$について、input(a)および$\Delta(a)$は未定義となる。
% また、$\tau = 0$のとき
% \begin{center}
%   $\tau - \Delta(\tau) = -\Delta(0) < 0$
% \end{center}
% である。
% この2点から、$\tau = 0$のとき時変値inputの直前値である$input(\tau - \Delta(\tau))$
% および時変値$\Delta$の直前値である$\Delta(\tau - \Delta(\tau))$は未定義となると分かる。
% また、現在値あるいは直前値が未定義となる場合が存在する時変値は、inputと$\Delta$の他に存在しない。
% このことから補題2が成り立つ。
% \begin{itembox}[l]{補題2}
%   $\tau = 0$のとき処理7の実行に成功する $\Longleftrightarrow$ \\
%   プログラム中に、時変値${\it input}$あるいは$\Delta$に対する直前値参照が存在しない。
% \end{itembox}
%
% 補題1および補題2から、プログラムの実行可能性について以下が成り立つ。
% \begin{screen}
%   プログラムが実行可能である $\Longleftrightarrow$ \\
%   プログラム中に、時変値${\it input}$あるいは$\Delta$に対する直前値参照が存在しない。
% \end{screen}
%
% このことから、時変値$input$および$\Delta$の初期値($t < 0$における値)が特別に定義されていれば、
% プログラムは必ず実行可能であるということも分かる。
