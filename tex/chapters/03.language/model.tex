本節では、具体的な構文には触れることなく、SFRPがどのようなプログラミング言語であるかについての解説を行う。
まず\ref{sec:language:model:program}項において、SFRPがどのようにシステムを表現するプログラミング言語であるのかを説明する。
続く\ref{sec:language:model:execution}項では、SFRPプログラムがどのように実行されるのかを示す。
さらに\ref{sec:language:model:validation}項において、プログラムの実行可能性について述べる。

\subsection{プログラムの表現}\label{sec:language:model:program}
SFRPプログラムは、時刻によって変動する値(時変値、time-varying value)によって構成される。
一つのSFRPプログラムは、入力時変値を前提とした出力時変値の導出であると言える。

以下に、SFRPにおいて時変値がどのように表現され、ユーザはどのような形で時変値を定義することができるのかを述べていく。

\begin{itemize}
  \item
  時変値は時刻$t$をパラメータに取る関数とみなされる。
  \begin{equation*}
    Time \rightarrow Value
  \end{equation*}

  \item
  時刻$t$におけるプログラムへの入力は、プリミティブな時変値
  \footnote{他の時変値を用いて定義されるのではなく、前提として与えられる時変値をプリミティブな時変値と呼ぶ。}
  として表現される。
  \begin{equation*}
    {\it input} \, :: \, Time \rightarrow Value
  \end{equation*}

  \item
  他の時変値を参照して演算を加えることにより、新たな時変値を定義することができる。
  たとえば、時変値Aの値と時変値Bの値の和を取る時変値Cを定義することができる。
  \begin{equation*}
    C(t) := A(t) + B(t)
  \end{equation*}

  \item
  各時変値は、時刻$t$における値(現在値)の参照と、微小時間$ \Delta t(t) $前の値(直前値)の参照が許されている。
  たとえば、時変値Aの現在値と時変値Bの直前値の和を取る時変値Dを定義することができる。
  \begin{equation*}
    D(t) := A(t) + B(t - \Delta t(t))
  \end{equation*}

  \item
  $ \Delta t $は微小時間を表すプリミティブな時変値である。
  \begin{equation*}
    \Delta t(t) > 0
  \end{equation*}

  \item
  初期値($ t < 0 $における値)を特別に指定した時変値を定義することができる。
  たとえば、$t \geq 0$においては時変値Aの値を取り、$t < 0$においては0を取る時変値Bを定義することができる。
  \begin{equation*}
    B(t) := \begin{cases}
      A(t) & (t \geq 0) \\
      0 & (t < 0)
    \end{cases}
  \end{equation*}

  \item
  時刻$t$におけるプログラムの出力は、任意の時変値によって定義される。
  以下は、時変値Dの現在値に1を加えた時変値を出力とする例である。
  \begin{equation*}
    output(t) := D(t) + 1
  \end{equation*}
\end{itemize}

\subsection{プログラムの実行}\label{sec:language:model:execution}
\ref{sec:language:model:program}項にて示したプログラム表現は、 以下に示す逐次的操作によって実行される。
\begin{screen}
\begin{enumerate}
  \item $ {\it timestamp}^{\prime} := 現在時刻$
  \item $ {\it timestamp} := 現在時刻 (> timestamp^{\prime})$
  \item $ {\it elapsed} := {\it timestamp} - {\it timestamp}^{\prime} $
  \item $\tau$が未定義であるなら$ \tau := 0 $として、そうでないなら$ \tau := \tau + {\it elapsed} $とする。
  \item $ \Delta t(\tau) := {\it elapsed} $
  \item 入力を受け取り、$ {\it input}(\tau) $の値とする。
  \item プログラム中の全ての時変値の、$t=\tau$における値を導出する。
  \item $ {\it output}(\tau) $の値を出力する。
  \item $ {\it timestamp}^{\prime} := {\it timestamp} $
  \item 2に戻る。
\end{enumerate}
\end{screen}

\subsection{プログラムの実行可能性}\label{sec:language:model:validation}
\ref{sec:language:model:execution}項の操作において実行可能性が明白でないのは、7番目の処理(処理7)である。
ただし入出力の処理は必ず成功すると仮定する。
本項では、処理7が実行可能である(すなわちプログラムが実行可能である)ための条件について議論を行う。

${\it output}(t)$の値の導出過程において、存在そのものが未定義な時変値が参照される場合は、当然の事ながら処理7は実行不可能である。
以下では、存在そのものが未定義な時変値への参照はプログラム中に含まれないと仮定する。
また、時変値の参照による依存関係に循環は無いと仮定する。

$t = \tau (\neq 0)$において処理7が実行される場合、
過去の$t = \tau - \Delta t(\tau)$である場合において処理7の実行に成功しているはずである。
したがって、プログラム中の任意の時変値$v(t)$について、$v(\tau - \Delta t(\tau))$の値は導出可能である。
このことから帰納的に定理1が成り立つ。
\begin{itembox}[l]{定理1}
  $\tau = 0$のとき処理7の実行に成功する $\Longleftrightarrow$ 常に処理7の実行に成功する
\end{itembox}

$\Delta t(0) > 0$であるから、$input(0-\Delta t(0))$および$\Delta t(0-\Delta t(0))$の値は未定義である。
また、時刻$-\Delta t(0)$における値が直接的に未定義となる時変値はこの2つを除いて存在しない。
このことから定理2が成り立つ。
\begin{itembox}[l]{定理2}
  $\tau = 0$のとき処理7の実行に成功する $\Longleftrightarrow$ \\
  プログラム中に、${\it input}(t - \Delta t(t))$あるいは$\Delta t(t - \Delta t(t))$への参照が存在しない。
\end{itembox}

定理1および定理2から、プログラムの実行可能性について以下の定理が成り立つ。
\begin{screen}
  プログラムが実行可能である $\Longleftrightarrow$ \\
  プログラム中に、$input(t)$あるいは$\Delta t(t)$への直前値参照が存在しない。
\end{screen}

このことから、時変値$input(t)$および$\Delta t(t)$の初期値($t < 0$における値)が特別に定義されていれば、
プログラムは必ず実行可能であるということも分かる。
