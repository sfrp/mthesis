\chapter{序論}
\section{研究の目的と背景}
非同期的に発生するイベントに反応し何らかの処理を行うことは、組込みシステムにとって典型的な動作である。
しかしながら、こういった反応的な処理は並行処理の機構を持たない手続き的なプログラミング言語においては簡潔に表現しづらい。
手続き的なプログラミング言語において反応的な処理を表現する場合、
入力値の取得と出力値の更新を順に行う処理をループさせる手法を用いるのが最も直接的ではあるが、
冗長で複雑な記述になりがちである。
割り込みやソフトウェアスレッドなどを利用してコールバック処理を行うという手法も考えられるが、
根本的な解決とはならない場合が多い。
コールバック処理はコンポーネント化が難しく、
複数の処理を組み合わせたり既存の処理を拡張したりといった事を行うに連れて記述が複雑になっていくためである。
また、非同期的にコールバック処理が実行されることによって却ってプログラムの見通しが悪くなることも多い。

このような反応的な処理を簡潔に記述する方法として、
現在では関数リアクティブプログラミング(Functional Reactive Programming, FRP)や
リアクティブプログラミング(Reactive Programming, RP)等と呼ばれる手法が広く利用されるようになってきた。
Webフロントエンド向けのElm言語\cite{czaplicki2012elm,czaplicki2013asynchronous}や
Reduxフレームワーク\cite{reduxwebsite}、
様々なプログラミング言語において実装されているReactiveXライブラリ\cite{rxwebsite}などがこれに該当する。

しかし、マイクロコントローラを始めとするメモリ制約の厳しい環境に対して、従来のFRPツール
\footnote{ここではFRP言語、FRPライブラリ、FRPフレームワーク等を総称してこう呼んでいる。}
をそのまま利用することは難しい。
従来のFRPツールはJavaやHaskell等のメモリを多く消費するプログラミング言語を前提とするものが大半であり、
それらをマイクロコントローラ上で実行することは難しいためである。

本研究では、マイクロコントローラにおいて実行可能な独自のFRP言語を設計及び実装する。
この言語は、従来研究されてきたFRPの手法に大枠では従いつつも、
小規模組込みシステムを記述するにあたって最適となるように設計した新たなFRPの手法を取り入れており、
省メモリ環境において実行時エラーを引き起こすことなく安全に実行可能であることを念頭に置いている。

\section{本論文の構成}
本論文では、始めに背景知識としてFRPとマイクロコントローラについて説明する。
第3章では本研究において独自に設計したFRP言語のデザインについて説明する。
第4章では本研究において実装したこのFRP言語のコンパイラついて説明する。
第5章ではこのコンパイラを用いて実際に実用的な小規模組込みシステムの開発を行い、
その過程を示すと共に完成品の性能評価を行う。
第6章では本言語の記述性および実用性について議論し、続く第7章では関連研究を交えた議論を行う。
最後に第8章にて本研究の結論を述べる。
